\chapter{Single-channel SAR}


\section{Textures}

%---------------------------------------------------------------
\subsection{Variation coeffiecient}
%---------------------------------------------------------------

This function calculates the variation coefficient of an
amplitude or intensity SAR image. The variation coefficient
is defined as
%\be
\begin{equation}
  var = \frac{variance(x)}{mean(x)^2}
\end{equation}
%\ee
and is a measure of the amount of local gray-value variation.
In homogeoneous regions, its value is idependent of the 
local image amplitude; edges appear emphasized.

RAT allows to set the size in $x$ and $y$ of the box used
for estimation of variance and mean.

%---------------------------------------------------------------
\subsection{RFI filter}\label{sec:sarrfi}
%---------------------------------------------------------------

This routines (tries) to filter out so-called radio frequency interferences (RFI).
RFI come from received energy, which was not emitted by the SAR (mobile phones,
radio beacons, etc).

You have to set precisely some system parameters. RAT assumes the near range
to be on the left side of the displayed image! That's important. If the system
parameters are not precise, you can expect very stange filtering results. 

After starting the detection and waiting a bit, an image is shown, where potential RFI appear as
"white" spots. You can now use the sliders to adjust the notch filter. Ideally,
it should cancel out all the white spots (but not the rest). There are two filter parameters: The detection threshold sets the sensitivity
of detecting interferences. A value of 2.0 means that all RFI having an energy
of larger then 2.0 times the background are detected. The filter strength sets
the radius of the notch filter used for removing the detected RFI. A value of
3 means that a circle of radius 3 around the RFI peak is removed. You'll have to
press 'update window' if you want to see what'll happen when using the actual
filter parameters.

When you think your settings are fine, press "start filtering"....

\subsection{Co-occurence features}

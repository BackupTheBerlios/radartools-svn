\chapter{Development Guide}

This section is meant as a reference for the developers of RAT, as well as for
those of you, who would like to understand better RAT was programmed.


\section{Import Template}

\section{Data Parameter Handling}
This section answers the questions:
\begin{itemize}
  \item How can I get access to parameters, resp. read/write them?
  \item How do I include new parameters.
  \item Where should I get the real parameters from?
\end{itemize}

\begin{enumerate}
  \item Define the parameter in the parameter structure \textbf{parstruct} in
    \textbf{definitions.pro}. The name of the entry is also the name of the
    parameter, followed by an NIL--Pointer type. Don't forget to add some
    comments about your parameter. Also think of a good name!
  \item For changing a parameter use the function \textbf{set\_par()}. You have to
    provide the name as a string and the value(can also be an array). As a
    result you get a status flag, which you should check. If you get 0, then
    everything is ok.
  \item For getting a parameter use the function \textbf{get\_par()}. See details above.
\end{enumerate}


\section{Speckle Filter Wizard}
This section answers the questions:
\begin{itemize}
  \item How do I include my newly written speckle filter into the Wizard.
  \item I have changed some parameters in my speckle filter. What do I have to
    modify in the Wizard?
\end{itemize}
Following modifications need to be considered:
\begin{enumerate}
  \item Put the name of the filter into the string--array
    \textbf{filter\_type}. Remember the position of your filter (index in the array).
  \item In dependence of the data type (POLINSAR, POLSAR, others) you should
    either \emph{offer} or \emph{not offer} this speckle filter to use. Set the right
    flag in the byte--array \textbf{offer}.
  \item Write generic fields for input in the widget at the right position.
  \item Call you speckle filter with the right parameters. Compere the fields
    you defined for the widget.
\end{enumerate}

\paragraph{Example}

\begin{enumerate}
  \item
\begin{verbatim}
  filter_type = [...,'MY-FILTER',...]  ; at position Z
\end{verbatim}
  \item
\begin{verbatim}
   POLINSAR --> offer = [...,1,...]  ; at position Z
   POLSAR   --> offer = [...,0,...]  ; at position Z
   OTHERS   --> offer = [...,0,...]  ; at position Z
\end{verbatim}
   with it this filter is allowed only for POLINSAR data
  \item 
\begin{verbatim}
Z: begin                ; MY-FILTER
   field[0,i]=CW_FIELD(filter_grp[i],VALUE=50,/integer,TITLE='Parameter1 : ',XSIZE=3)
   field[1,i]=CW_FIELD(filter_grp[i],VALUE=25,/integer,TITLE='Parameter2 : ',XSIZE=3)
\end{verbatim}
   \item
\begin{verbatim}
;; MY-FILTER
   Z: my-filter,/called,par1=*field_value[0],par2=*field_value[1]
\end{verbatim}
   \end{enumerate}
And that's it!

\chapter{Developer's Guide}

This section is meant as a reference for the developers of RAT, as well as for
those of you, who would like to understand better how RAT was programmed and how
to modify it by yourself.


\section{Import Template}

\subsection{Block Processing (Tiling)}

\section{Data Parameters Handling (Metadata)}
This section answers the questions:
\begin{itemize}
  \item How can I get access to parameters, resp. read/write them?
  \item How do I include new parameters.
  \item Where should I get the real parameters from?
\end{itemize}

\begin{enumerate}
  \item Define the parameter in the parameter structure \textbf{parstruct} in
    \textbf{definitions.pro}. The name of the entry is also the name of the
    parameter, followed by an NIL--Pointer type. Don't forget to add some
    comments about your parameter. Also think of a good name!
  \item For changing a parameter use the function \textbf{set\_par()}. You have to
    provide the name as a string and the value(can also be an array). As a
    result you get a status flag, which you should check. If you get 0, then
    everything is ok.
  \item For getting a parameter use the function \textbf{get\_par()}. See details above.
\end{enumerate}


\section{Speckle Filter Wizard}
This section answers the questions:
\begin{itemize}
  \item How do I include my newly written speckle filter into the Wizard.
  \item I have changed some parameters in my speckle filter. What do I have to
    modify in the Wizard?
\end{itemize}
Following modifications need to be considered:
\begin{enumerate}
  \item Put the name of the filter into the string--array
    \textbf{filter\_type}. Remember the position of your filter (index in the array).
  \item In dependence of the data type (POLINSAR, POLSAR, others) you should
    either \emph{offer} or \emph{not offer} this speckle filter to use. Set the right
    flag in the byte--array \textbf{offer}.
  \item Write generic fields for input in the widget at the right position.
  \item Call you speckle filter with the right parameters. Compere the fields
    you defined for the widget.
\end{enumerate}

\paragraph{Example}

\begin{enumerate}
  \item
\begin{verbatim}
  filter_type = [...,'MY-FILTER',...]  ; at position Z
\end{verbatim}
  \item
\begin{verbatim}
   POLINSAR --> offer = [...,1,...]  ; at position Z
   POLSAR   --> offer = [...,0,...]  ; at position Z
   OTHERS   --> offer = [...,0,...]  ; at position Z
\end{verbatim}
   with it this filter is allowed only for POLINSAR data
  \item 
\begin{verbatim}
Z: begin                ; MY-FILTER
   field[0,i]=CW_FIELD(filter_grp[i],VALUE=50,/integer,TITLE='Parameter1 : ',XSIZE=3)
   field[1,i]=CW_FIELD(filter_grp[i],VALUE=25,/integer,TITLE='Parameter2 : ',XSIZE=3)
\end{verbatim}
   \item
\begin{verbatim}
;; MY-FILTER
   Z: my-filter,/called,par1=*field_value[0],par2=*field_value[1]
\end{verbatim}
   \end{enumerate}
And that's it!


\section{Batch processing with RAT}

This section describes the possibilities of batch processing with RAT. The
batch processing should provide the possibilities for

\begin{itemize}
  \item fast data processing from the shell without graphical output (which can
  be very time consuming in case of big data sets).
  \item writing of IDL batch
  programs for automatic and fully operational SAR data processing with RAT
  procedures.
  \item shell-only radar data processing, without any mouse.
\end{itemize}

\subsection{Basic Concepts}
General processing chain:
\begin{enumerate}
  \item Open IDL shell
  \item Compile RAT
\begin{verbatim}
  .compile rat
\end{verbatim}
  \item Start RAT in the batch mode with
\begin{verbatim}
  rat, /nw ;;   nw for NoWindow
\end{verbatim}
  \begin{itemize}
    \item As usual with RAT, if a data set with the name "default.rat" can be
    found in the working directory, RAT opens it automatically.
    \item Alternatively, one can provide a file name to open at the start with
\begin{verbatim}
    rat, /nw, startfile="my_data.rat"
\end{verbatim}
  \end{itemize}
  \item One can open a new data set with
\begin{verbatim}
  open_rat, inputfile="my_data.rat"
\end{verbatim}
  \item And, similarly, one can save the current working data
  set with
\begin{verbatim}
  save_rat, outputfile="my_modified_data.rat"
\end{verbatim}
  \item The current data set can
  also be exported to an IDL variable, and used further without interaction with
  RAT: 
\begin{verbatim}
  export_var, variable_name
\end{verbatim}
  \item At any moment, the GUI can be shown, and hidden
  again, which makes it possible to switch between batch processing, and the
  graphical representation: 
\begin{verbatim}
  show_gui
  hide_gui
\end{verbatim}
  \item To exit RAT (batch processing or
  GUI), write
\begin{verbatim}
  exit_rat
\end{verbatim}
\end{enumerate}

In a similar fashion, one can import data from external file formats, apply
various processing functions, and export it again.

It is of importance to provide all the required parameters to the functions. To
find out, which parameters these are, please see this documentation, or consult
the IDL source codes of the functions.

It is also important to use the keyword "/CALLED" when calling procedures from
the shell, which act on the data.

Note: This mode is still at an experimental stage. Therefore, it is possible
that for some untested procedures, some windows will pop up. Also, some
functions are not recommened to use in the batch mode, for instance, there is no
reason to call a zoom procedure within the batch processing.

In the following, two examples of RAT batch scripts are presented:

1) Simple steps to open a PolSAR data, to do speckle filtering, and to decompose
into Entropy/Alpha/Anisotropy. The data is then exported to an IDL variable,
which can be further examined, even after exiting RAT.


\begin{verbatim}
  .compile rat		;; compile RAT
  rat, /nw, startfile="some_polarimetric_C_or_T_matrix_data.rat"
  speck_polreflee, smm=7, looks=9, /CALLED
  decomp_entalpani, /CALLED
  show_gui	;; just to see, what did come out
  hide_gui	;; hide the GUI again
  export_var, HAA  	;; export the data into the given variable "haa"
  exit_rat	;; quit rat batch processing
  help,HAA  	;; shows the structure of the still existent variable "haa"
\end{verbatim}

2) Advanced automatic processing: A function to construct a MB PolInSAR data set
from a given set of PolSAR data files. Chain flow: construct the MB data set,
remove the flat earth phase, conduct adaptive range spectral filtering,
construct a coherency matrix, apply simultaneously a specified presuming, and do
speckle filtering with the Refind-Lee filter. The file is finally saved in the
rat format.


\begin{verbatim}
  pro mbpolinsar_data_processing, polsar_files, fe_file, outputfile, $
                                  smmx=smmx, smmy=smmy, Lee_box=box
    rat,/nw
    construct_polinsar, /CALLED, FILES=polsar_files
    rrat, fe_file, fe
    polin_rgflt_adaptive, /CALLED, fe
    polin_k2m, /CALLED, SMMX=smmx,SMMY=smmy
    polin_c2t
    speck_polreflee, /CALLED, SMM=Lee_box, LOOKS=smmx*smmy
    save_rat,outputfile=outputfile
    exit_rat
  end
\end{verbatim}

This function can then be used for automatic data processing, e.g. with:
\begin{verbatim}
  .compile rat
  mbpolinsar_data_processing, ["track1.rat","track2.rat","track3.rat"], $
      "fe_1x2x3.rat", "mb_tracks_1x2x3.rat", smmx=3, smmy=9, Lee=7
\end{verbatim}
These two examples should only show the possibilities of batch processing with RAT.


\section{Mini Howto on RAT-SVN via Linux shell}
Attention! To have a direct access to the subversion repository, as desribed in
the following, you need the status of a RAT--developer on the server.

Checkout of the whole trunk from the server to the current directory:
\begin{verbatim}
  svn checkout https://YOUR_NICK@svn.berlios.de/svnroot/repos/radartools/trunk
\end{verbatim}
The following operations can be executed in the trunk directory or any
subdirectory.

Current status:
\begin{verbatim}
  svn status -u
\end{verbatim}
Update the local copy from the server:
\begin{verbatim}
  svn update
\end{verbatim}
Commit your changes (do ALWAYS add useful comments to the changes after $-m$):
\begin{verbatim}
  svn commit -m "Change Log: ..."
\end{verbatim}


\section{Building a Binary for the Release}
IDL:
\begin{verbatim}
  .compile rat
  resolve_all
  save,filename="rat.sav",/routines
\end{verbatim}
Pack together with \emph{icons} and \emph{preferences} into a single archive.
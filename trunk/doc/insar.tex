\chapter{SAR Interferometry (InSAR)}
RAT can handle interferometric image pairs. An interferometric image
pair is stored in a single file, in pixel interleave format. 

%---------------------------------------------------------------
\section{Coregistration}
%---------------------------------------------------------------
Interferometric image pairs are usually not coregistered. Reasons
can be the baseline between the images, different range delays, shifted
image frames etc. As even a small misregistration drastically degrade the
coherence of an interferogram, before further processing a precise
coregistration has to be performed.

%---------------------------------------------------------------
\subsection{Global offset}\label{sec:insar_goff}
%---------------------------------------------------------------
This routine can shift the whole slave image by a constant offset.
The offset has to be an integer number, i.e. subpixel shifts
can not be performed with this routine. Enter the offset values in 
the upper box and press 'OK' to shift the slave image. 

If you don't know the image offset, you can use the "automatic offset
estimation" to estimate it. The default values in the boxes should be
o.k. for most cases. You have to change them only if you have images
with very large offsets. Shortly after "start estimation" has been pressed,
the estimated values appear in the upper box. Press "OK" to shift
the slave.

For offset estimation, the method of amplitude correlation is used. 
%---------------------------------------------------------------
\subsection{Subpixel offset}\label{sec:insar_soff}
%---------------------------------------------------------------
Very much the same as function \ref{sec:insar_goff}. A constant shift
of the entire slave image is performed, this time allowing floating point
values as offsets. For shifting, cubic interpolations are used.
It is not necessary to use this function when it is planned to use
"array of patches" or similar afterwards.

The estimation is based on oversampled amplitude correlation is used.
It is not possible to estimate very large offsets. In this case use
function \ref{sec:insar_goff} before.
%---------------------------------------------------------------
\subsection{Array of patches}\label{sec:insar_soff}
%---------------------------------------------------------------
This function estimates image offsets on a grid of small patches
distributed over the image. Then, a 2D-warping function is applied on
the slave image for coregistration. This function is useful if the
offsets are not constant over the image, which is often the case
in airborne repeat-pass images or spaceborne images with strong
topography.

In the first box, the number of patches and their size can be specified.
Don't select a too small size of the patches, since this will make the
results unreliable. Press "OK" to start the estimation.

After a while, a second box will appear showinh the vector field of
offsets. With the 2 edit buttons you can eliminate by hand obviously
wrong estimates. When everything looks good, press "Start coregistration"
to warp the slave image (time consuming!). Oversampling factors of 2 or 4
can help to better preserve image quality.

For offset estimation, oversampled amplitude correlation is used. For
warping, cubic convolution, is desired combined with FFT based 
oversampling, is used.

%---------------------------------------------------------------
%---------------------------------------------------------------
\section{Phase noise filter}
%---------------------------------------------------------------
%---------------------------------------------------------------
Phase noise filtering is used to enhance the quality of the interferometric
phase. It is usually employed prior to phase unwrapping, but can also be
useful for other purposes.
%---------------------------------------------------------------
\subsection{Boxcar}
%---------------------------------------------------------------
Boxcar filtering is a very simple filter: It is just a 
local averaging over the give rectangular box. Becuase of phase
wrapping effects, averaging is performed in the complex domain,
even if only a floating point phase is provided.
%---------------------------------------------------------------
\subsection{Goldstein}
%---------------------------------------------------------------
The Goldstein filter is a spectral filter which enhances
dominant fringe components in a local box. This is achieved by
multiplying the fringe spectra with its own amplitude power a
certain filter parameter (spectral exponent). Goldstein filtering
is very powerful in relatively high coherent regions, but might 
fail in low coherent areas.

In RAT, the window size is fixed to 64x64 pixels with an overlap
of 32 pixels.

%---------------------------------------------------------------
\subsection{GLSME}
%---------------------------------------------------------------
Another InSAR specific fringe filter. GLSME filtering minimises
globally the phase noise fluctuations by a least-squares approach.
GLSME seems to be very powerful especially for very low
coherent regions surrounded by high coherent areas.


%---------------------------------------------------------------
%---------------------------------------------------------------
\section{Coherence}
%---------------------------------------------------------------
%---------------------------------------------------------------
\subsection{Boxcar}
\subsection{Gauss}
\section{Remove flat-earth}
\subsection{linear}
\subsection{from file}
\section{Shaded relief}
\section{Transform}
\subsection{pair to interferogram}
\subsection{extract amplitude}
\subsection{extract phase}
